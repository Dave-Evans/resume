\documentclass{article}
\usepackage{mathtools}
\usepackage{setspace}
\usepackage{textcomp}
\usepackage{hyperref}
\pagenumbering{gobble}
\pagestyle{myheadings}
\markright{ Resume David Evans \hfill \today}
\usepackage[margin=0.34in]{geometry}
%\setlength{\topmargin}{-.5in}
%\setlength{\textheight}{9in}
%\setlength{\oddsidemargin}{.110in}
%\setlength{\textwidth}{7in}
\begin{document}
\begin{centering}   
\Huge{David Evans} \\
\large{  3704 36th Avenue S, Minneapolis, MN 55406 \\
651-528-0375 $|$ evans.dave.michael@gmail.com} \\
(\href{https://github.com/Dave-Evans}{https://github.com/Dave-Evans}) \\ 
\bigskip

\onehalfspacing
\end{centering}


\noindent \large{OBJECTIVE}\\
\line(1,0){580}\\
I am a data engineer who is looking for opportunities to apply my current toolset as well as learn new ways of thinking.
I am a curiosity-driven tinkerer who loves developing new solutions and finding ways to make existing processes more efficient.
% For me, there is nothing like the thrill of solving a new problem and figuring out difficult issues.
I have become adept at finding ways to work with clients to solve business problems while balancing the perfect with the good.
% While I love developing new solutions, I love getting absorbed in finding ways to make existing processes more efficient.
 \\

\noindent \large{SKILL SET}\\ 
\line(1,0){580}\\
\parbox[t][3cm][t]{0.4\textwidth}{
	\textbullet Data manipulation, engineering, and munging\\
	\textbullet Process automation\\
	% \textbullet Spatial data analysis, manipulation, and management\\
	% \textbullet Statistical analysis and inference\\
	\textbullet Application development\\
	\textbullet GIS theory and practice\\
	\textbullet Web development\\

	% \textbullet Spatial statistics, including kriging, point pattern analysis, and regression\\
	% \textbullet Field description of soils\\
	% \textbullet Laboratory analysis and procedures\\
	% \textbullet Written and oral communication
} 
\parbox[t][2.5cm][t]{0.07\textwidth}{\hfil}
\parbox[t][3.2cm][t]{0.45\textwidth}{
	Software proficiencies and familiarities:\\
	\small{
		\textbullet R and Python\\
		\textbullet SQL\\
		\textbullet AWS cloud services\\
		\textbullet ArcGIS, QGIS, and SAGA-GIS\\
		\textbullet C\# and VB.NET\\
		\textbullet JavaScript\\
		\textbullet Git version control software\\
		\textbullet \LaTeX{} and Markdown\\
		\textbullet Bash and CMD\\
		% \textbullet SWAT and ArcSWAT watershed modeling software\\
		% \textbullet GDAL/OGR spatial data libraries
		% \textbullet ASP.NET MVC and Django web development frameworks\\		
		% \textbullet GPS equipment and methodologies\\
		% \textbullet Classification according to US Soil Taxonomy
		% \textbullet The typesetting software \LaTeX\\
	}
}

\noindent \large{EDUCATION}\\
\line(1,0){580}\\ \normalsize
\textbf{University of Wisconsin-Madison}\\
Master of Science, Soil Science \hfill August 2013\\ %; GPA: 3.719
% \indent Concentration: spatial analysis and pedology\\
\noindent Bachelor of Science, Soil Science \hfill May 2011\\ %; GPA: 3.714
% \indent Concentration: field crop production\\



\noindent \large{RELEVANT EXPERIENCE}\\ \begin{small}
\line(1,0){580}\\

	\noindent \textbf{Statistical Research Modeler -- American Family Insurance} \hfill \textbf{July 2016 -- Present}
	\begin{itemize}
		\item Data pipeline development to support company-wide modeling projects
		\item Develop agile reporting workflow to distribute customer information to agents
		\item Facilitate automation system for mundane tasks
		\item Implement version control practice on data science team
	\end{itemize}
	


	% ----------------------------------------------------------- %
	\noindent \textbf{GIS Analyst and Developer -- Legislative Technology Services Bureau} \hfill \textbf{May 2015 -- June 2016}
	\begin{itemize}
		\item Web site development and maintenance
		\item Developed desktop applications using ArcObjects for .NET
		\item Assisted in ArcGIS Server maintenance and updates
	\end{itemize}
	% ----------------------------------------------------------- %
	\noindent \textbf{IS Data Services -- Department of Natural Resources} \hfill \textbf{July 2014 -- May 2015}
	\begin{itemize}
		\item Assemble and organize spatial and non-spatial data for large watershed modeling project
		\item Carry out statistical analysis for the regionalization of hydrologic model parameters
		\item Assist with the development of a spatial analysis tool 
		 for predicting erosion and improving water quality
	\end{itemize}
	% ----------------------------------------------------------- %
%	\noindent \textbf{GIS Technician -- Natural Resources Conservation Service} \hfill \textbf{January 2014 -- June 2014}
%	\begin{itemize}
%		\item Assist in the maintenance of wetland determinations database.
%		\item Conduct orthorectification and wetland map creation. 
%	\end{itemize}
	% ----------------------------------------------------------- %
%	\noindent \textbf{GIS Analyst -- UW-Madison Soil Science, Nutrient Cycling and Agroecosystems Lab} \hfill \textbf{October 2013 -- June 2014}
%	\begin{itemize}
%		\item Facilitate the investigation of fine-scale soil variation in a fertilizer management study.
%		\item Conduct spatial analysis of interaction of nitrogen rate and soil property in relation to crop yield.
	%\item Carry out statistical anaysis to determine where and how fertilizer can be applied most profitably.
%	\end{itemize}
	% ----------------------------------------------------------- %
	\noindent \textbf{Data Coordinator -- UW-Madison Soil Science, F.D. Hole Soils Lab} \hfill \textbf{August 2013 -- June 2014}
	\begin{itemize}
		\item Coordinate the progress and maintenance of map display and development for Wisconsin branch of the ISEE Project
		\item Develop maps of landscapes and soil properties for educational purposes
		\item Organize and maintain database of geospatial information
	%\item Create workflow for map creation and revision.
	%\item Investigate workflow for managing revisions and updates to published maps.
	\end{itemize}
	% ----------------------------------------------------------- %
	\noindent \textbf{Research Assistant -- UW-Madison Soil Science, F.D. Hole Soils Lab} \hfill \textbf{Sept 2011 -- June 2013}
	\begin{itemize}
		\item Used digital soil mapping techniques to produce quantitative maps of soil properties
		\item Described and classified soils in the field according to US Soil Taxonomy
		\item Investigated soil-landscape relationships through statistical and conceptual models
	\end{itemize}
	% -------------------- %
	%\noindent \textbf{Field Assistant -- UW-Madison Soil Science, Nutrient Cycling and Agroecosystems Lab} \hfill \textbf{March 2010 -- Aug 2011}
	%\begin{itemize}
	%\item Managed analysis of soil and water samples; compiled data for multi-site, multi-year study.
	%\item Prepared and maintained research plots with field technicians.
	%\item Collected, catalogued, and processed soil, water, biomass, and grain samples.
	%\end{itemize}
	% -------------------- %
	%\noindent \textbf{Investigator -- UW-Madison Soil Science, Bioretention Pond Permeability Project} \hfill \textbf{Fall 2010 -- Spring 2011}
	%\begin{itemize}
	%\item Investigated causes for the failure of a stormwater retention basin with a team of professors, graduate and undergraduate students.
	%\item Conducted laboratory analysis to test hypotheses regarding the soil properties contributing to basin failure.
	%\item Communicated potential solutions to hydrologists and urban planners.\\
	%\end{itemize}
	% -------------------- %
	%\noindent \textbf{Participant -- NACTA Soil Judging Contest} \hfill \textbf{Spring 2011}
	%\begin{itemize}
	%\item Silver medal in individual category of national soil judging competition against fifty students from nine colleges and universities.
	%\item Identification, description, and classification of soils in the field.
	%\item As captain, led team members in weekly meetings and practices.\\
	%\end{itemize}
	% -------------------- %
\end{small}

% \noindent \large{SELECTED PRESENTATIONS \& PUBLICATIONS}\\ 
% % % %
% \line(1,0){580}\\
% 
% \begin{small}
% \leftskip 0.1in
% \parindent -0.1in
	% -------------------- % 

% Ruesch, A. S., A. Mynsberge, and D. M. Evans. ``Introduction to Python in ArcGIS.'' 10 February, 2016. A workshop given at the Wisconsin Land Information Association in Elkhart Lake, WI.\\

	% -------------------- % 

%Ruesch, A. S., A. Mynsberge, and D. M. Evans. ``Introduction to Python in ArcGIS.'' 18 February, 2015. A workshop given at the Wisconsin Land Information Association in Green Bay, WI.\\

	% -------------------- % 
 
% Evans, D. M. and A. E. Hartemink. ``Digital soil mapping of a red clay subsoil covered by loess.'' 2014. Geoderma 230-231, 296-304.\\

	% -------------------- % 

%Evans, D. M. and A. E. Hartemink. ``Digital Soil Mapping of Terra Rossa in Southern Wisconsin.'' 22 October, 2012. Presentation given at the New %Challenges for Digital Soil Mapping: I, session at the ASA, CSSA and SSSA Annual Meetings in Cincinnati, OH.\\

	% -------------------- % 

%Evans, D. M. and A. E. Hartemink. ``Mapping the Presence of Red Clay Subsoil in the Driftless Area of

	% -------------------- % 

%Wisconsin, USA.'' \textit{Digital Soil Assessments and Beyond: Proceedings of the 5th Global Workshop on Digital Soil Mapping 2012}, Sydney, Australia. Minasny, B., Malone, B. P., and McBratney, A. B. CRC Press. July, 2012.\\

% Evans, D. M., K. Rudersdorf, G. Obear, S. Griffith, M.R. Naber, and P. Barak. ``Saline and Sodic Soils in Wisconsin: The Sad Story of the Costco Infiltration Pond.'' 12 September, 2012. Presentation given to USGS Water Resources Meeting, Madison, WI.\\


% \end{small}

\end{document}
